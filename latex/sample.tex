\documentclass{jcse}

\usepackage{url} % generates hyperlinks if available + modifies _ in urls.
\usepackage{amsmath} % higher quality rendering of eqs
\usepackage{flushend}
% flushend package is the easiest way to balance the two
% columns at "references" section.
\usepackage{booktabs} % higher quality tables
\usepackage{algorithm} % provides algorithm env.
\usepackage{algpseudocode}
\papertype{Regular Paper}
\papertypeAtFoot{Open Access}
\volumeyear{Jan 0000} \volumenumber{00} \issuenumber{0} \pp{000-000}
\DOI{yy.5626/JCSE.2011.5.2.xxx}

\received{00 Month 2011}
\revised{00 Month 2011}
\accepted{00 Month 2011}

\permissionUrl{http://jcse.kiise.org}

\copyrightString{Copyright \textcopyright 2011. The Korean Institute of Information Scientists and Engineers}

\issn{pISSN: 1976-4677 eISSN: 2093-8020}

\permission{
This is an Open Access article distributed under the terms of the Creative Commons Attribution Non-Commercial License (\url{http://creativecommons.org/licenses/
by-nc/3.0/}) which permits unrestricted non-commercial use, distribution, and reproduction in any medium, provided the original work is properly cited.
}
 % Setup for editors. Paper authors don't need to modify this.

% For Title, Use Camel Case.
\title[JCSE Template Example]{An Example Usage of the JCSE \LaTeX\xspace Class}
\author{
	\authname{Debra Park} \\
	\affaddr{School of Electrical Engineering and Computer Science,
	  ABC National University, Seoul, Korea}
	\email{debbie@abc.ac.kr} \\
	\\
	\authname{Dezhen Zhu$^*$} \\
	\affaddr{School of Electrical Engineering and Computer Science,
	  ABC National University, Seoul, Korea}
	\email{jdj@abc.ac.kr} \\
}
\shortauthors{D. Park and D. Zhu}
\bottomstuff
{
$^*$ Corresponding Author
}

% Enable one of following six categories listed at
% http://jcse.kiise.org/
\category{Embedded Computing}
%\category{Ubiquitous Computing}
%\category{Convergence Computing}
%\category{Green Computing}
%\category{Smart and Intelligent Computing}
%\category{Human Computing}

% Any other keywords...
% Capital letter only for the first word of the first keyword.
\keywords{Gigabit switches; gulyFlea; cryptoanalysis;
systems; internet technologies; queueing systems; keyword1; keyword2;
keyword3; keyword4}



\begin{document}

\begin{abstract}
This is an example tex file which demonstrates the functionalities of
JCSE \LaTeX template.
Lengthening the abstract for testing purpose:
Blah blah blah blah blah blah blah blah blah blah blah blah.
Blah blah blah blah blah blah blah blah blah blah blah blah.
Blah blah blah blah blah blah blah blah blah blah blah blah.
Blah blah blah blah blah blah blah blah blah blah blah blah.
Blah blah blah blah blah blah blah blah blah blah blah blah.
Blah blah blah blah blah blah blah blah blah blah blah blah.
Blah blah blah blah blah blah blah blah blah blah blah blah.
Blah blah blah blah blah blah blah blah blah blah blah blah.
Blah blah blah blah blah blah blah blah blah blah blah blah.
Blah blah blah blah blah blah blah blah blah blah blah blah.
Blah blah blah blah blah blah blah blah blah blah blah blah.
Blah blah blah blah blah blah blah blah blah blah blah blah.
Blah blah blah blah blah blah blah blah blah blah blah blah.
Blah blah blah blah blah blah blah blah blah blah blah blah.
Gigabit switches  must work. After years of key research into
checksums, we confirm the simulation of cache coherence, which embodies
the confusing principles of cryptoanalysis. GulyFlea, our new
methodology for read-write information, is the solution to all of these
grand challenges.
\end{abstract}

\maketitle



\section{Introduction} \label{s:intro}

This is an exmple \LaTeX\xspace file demonstrating the
usage of the JCSE \LaTeX\xspace class file.
Skip to section \ref{s:ex} for example tables, figures, etc.
Other than that, the rest of the paper is gibberish generated
by SCIgen \cite{scigen}.
\LaTeX\xspace codes are worth looking at, though.

The analysis of the transistor has investigated IPv6, and current
trends suggest that the emulation of expert systems will soon
emerge\footnote{Foot note test 1}.
But,  despite the fact that conventional wisdom states that this
question is mostly fixed by the emulation of Web services, we believe
that a different approach is necessary.
In fact, few system
administrators would disagree with the construction of the transistor,
which embodies the private principles of cryptoanalysis \cite{cite:0}.
To what extent can IPv7  be constructed to fix this obstacle?

We concentrate our efforts on showing that the foremost low-energy
algorithm for the understanding of write-ahead logging by Sun
\cite{cite:1} is maximally efficient \cite{cite:2}.  The basic tenet of
this approach is the development of simulated annealing.  Existing
wireless and ``smart'' systems use 802.11b  to cache systems.  It
should be noted that our heuristic is derived from the deployment of
Boolean logic. Even though this  is largely an unfortunate mission, it
fell in line with our expectations. On a similar note, indeed, the
UNIVAC\footnote{Foot note test2} computer  and Lamport clocks  have a
long history of interfering
in this manner. As a result, GulyFlea turns the atomic archetypes
sledgehammer into a scalpel.

Our contributions are as follows.   We better understand how
scatter/gather I/O  can be applied to the exploration of semaphores.
Second, we argue that red-black trees  can be made game-theoretic,
Bayesian, and classical.

The roadmap of the paper is as follows. To begin with, we motivate the
need for the Internet.  We place our work in context with the related
work in this area.  We confirm the exploration of RAID. Next, we
confirm the construction of courseware. Finally,  we conclude.


\section{Examples} \label{s:ex}

\subsection{One Equation}

\begin{equation}
F = -\frac{GMm}{r^2}
\label{e:grav}
\end{equation}

Equations can be easily written in equation environment.
See Equation~\ref{e:grav} for example.


\subsection{Series of Equations}

\begin{eqnarray}
 y &=& x^4 + 4      \nonumber \\
   &=& (x^2+2)^2 -4x^2 \nonumber \\
   &\le&(x^2+2)^2 \label{e:series}
\end{eqnarray}

Sometimes a series of equations are needed.
See Equation~\ref{e:series} for
details\footnote{\url{http://www.personal.ceu.hu/tex/cookbook.html}}.


\subsection{Algorithms}

\begin{algorithm}
\caption{Euclid’s algorithm}\label{alg:euclid}
\begin{algorithmic}[1]
\Procedure{Euclid}{$a,b$}\Comment{The g.c.d. of a and b}
	\State $r\gets a\bmod b$
	\While{$r\not=0$}\Comment{We have the answer if r is 0}
		\State $a\gets b$
		\State $b\gets r$
		\State $r\gets a\bmod b$
	\EndWhile\label{euclidendwhile}
	\State \textbf{return} $b$\Comment{The gcd is b}
\EndProcedure
\end{algorithmic}
\end{algorithm}

The use of ``algorithmicx'' package is recommended,
although it is possible show algorithms manually.
An example is shown as Algorithm~\ref{alg:euclid}.


\subsection{Theorems, Proofs, etc.} \label{s:envs}

JCSE.cls profides many environments for theorems, proofs, etc,
Definitions \ref{def:3} and \ref{def:genus} are an example.
There are more environments:

\begin{definition}[A triangle]
\label{def:3}
A plane figure bounded by 3 straight sides is a triangle.
\end{definition}

\begin{definition}[a genus (or family)]
\label{def:genus}
An existing definition that serves as a portion of the new definition;
all definitions with the same genus are considered members of that genus,
and a definition can be composed of multiple genera (more than one genus).
\end{definition}

\begin{lemma}[Example Lemma]
Lemma lemma lemma lemma lemma lemma lemma lemma lemma.
\end{lemma}
\begin{proof}
Write proof here.
\end{proof}

\begin{corollary}[Example corollary]
Corollary corollary corollary corollary corollary.
\end{corollary}

\begin{proposition}[Example proposition]
All humans are mortal.
\end{proposition}

\begin{axiom}[Example Axiom]
It is possible to draw a straight line from any point to any other point.
\end{axiom}

\begin{remark}[Example Remark]
Can't prove axioms.
\end{remark}

\begin{example}[Example example]
There can be numbered examples.
\end{example}

\begin{theorem}[Fermat's Last Theorem]
No three positive integers $a$, $b$, and $c$ can satisfy the equation
$a^n + b^n = c^n$ for any integer value of $n$ greater than two.
\end{theorem}
\begin{proof}
I have discovered a truly remarkable proof which this margin
is too small to contain.
\end{proof}



\subsection{Subsection testing1}
Chicken chicken chicken chicken,
chicken chicken chicken chicken \cite{Zongker2006}.

\subsubsection{Subsubsection Testing1} \label{ss:testing1}

Subsubsection headings are enumerated by Arabic numerals followed by
parentheses. They are indented, italicized, upper and lower case,
run into the text in their sections, and are followed by a colon.

\subsubsection{Subsubsection Referencing}
It is possible to reference sections and subsections.
Subsubsection \ref{ss:testing1} has discussion about chickens.
Subsection \ref{s:envs} has examples of definitions, remarks, proofs, etc.
Section \ref{s:intro} is the introduction section.


\subsubsection{Subsubsection Image that Spans Two Columns}

\begin{figure*}
	\centering
	\epsfig{width=\textwidth,figure=figs/unix}
	\caption{Evolution of Unix and Unix-like systems. This is an example
		of large figures which spans two columns.
	}
	\label{dia:unix}
\end{figure*}

Fig.~\ref{dia:unix} is from \url{http://en.wikipedia.org/wiki/File:Unix_history-simple.svg}.

\subsection{Subsection testing2}
Chicken chicken chicken chicken,
chicken chicken chicken chicken.

\subsection{Example Tables}

\subsubsection{Simple Tables}

\begin{table}
\centering
\caption{Quarks}
\label{t:quarks}
\begin{tabular}{cccc}
\thline
Name     & Symbol & Antiparticle   & Charge (e) \\
\hline
up       & $u$    & $\overline{u}$ & +2/3       \\
down     & $d$    & $\overline{d}$ & -1/3       \\
charm    & $c$    & $\overline{c}$ & +2/3       \\
strange  & $s$    & $\overline{s}$ & −1/3       \\
top      & $t$    & $\overline{t}$ & +2/3       \\
bottom   & $b$    & $\overline{b}$ & -1/3       \\
\hline
\end{tabular}
\end{table}

``thline'' command provides thicker line for drawing tables,
as shown as Table~\ref{t:quarks}.


\subsubsection{(Optional) Booktabs Package}

\begin{table*}
\centering
\caption{Comparison of results by their algorithm \cite{Zongker2006} and our algorithm.}
\label{t:cmp}
\begin{tabular} {ccccccccc}

\toprule
        & \multicolumn{4}{c}{Theirs \cite{Zongker2006}}  & \multicolumn{4}{c}{Ours} \\
\cmidrule(r){2-5} \cmidrule(l){6-9}
Circuit & Skew (ps) & Hops & Latency (ps) & Error(\%)    & Skew (ps) & Hops & Time (ps) & Error(\%) \\
\midrule
s5378   & 40        & 3004 & 19.4         & 14.2         & 14.1      & 2990 & 12        & 0.02 \\
s9234   & 50        & 2345 & 23.6         & 18.3         & 18.2      & 2131 & 16        & 0.02 \\
s13207  & 134       & 2263 & 54.8         & 47.7         & 47.8      & 1950 & 36        & 0.19 \\
s15850  & 133       & 2615 & 56.7         & 51.5         & 52.9      & 1515 & 49        & 0.12 \\
s35932  & 407       & 8774 & 129.7        & 122.4        & 111.6     & 6684 & 121       & 1.26 \\
s38417  & 343       & 7917 & 113.7        & 100.1        & 112.8     & 7917 & 81        & 1.20 \\
s35584  & 330       & 268  & 119.8        & 113.1        & 99.8      & 255  & 100       & 0.80 \\
\cmidrule(lr){1-1} \cmidrule(r){2-5} \cmidrule(l){6-9}
Average & 3         & 4    & 5            & 6            & 7         & 8    & 9         & 10   \\
\midrule

\end{tabular}
\end{table*}

Table~\ref{t:cmp} shows an example of larger table which spans two columns.
``booktabs'' package allows higher quality tables.
Use toprule, midrule, bottomrule, cmidrule commands instead if using booktabs.
cmidrule command allows shorter version of ``clines'' as the two lines
between the first and second rows of Table~\ref{t:cmp}.



\section{Model}

The properties of our algorithm depend greatly on the assumptions
inherent in our framework; in this section, we outline those
assumptions. This may or may not actually hold in reality.  We
estimate that spreadsheets  can be made metamorphic, homogeneous, and
adaptive. This may or may not actually hold in reality.  Rather than
creating fiber-optic cables, GulyFlea chooses to request stable
theory. Therefore, the framework that GulyFlea uses is unfounded.

\begin{figure}
	\centering
	\epsfig{figure=figs/dia0}
	\caption{GulyFlea's encrypted location.}
	\label{dia:label0}
\end{figure}

Our heuristic relies on the structured architecture outlined in the
recent little-known work by J. Dongarra in the field of robotics. This
seems to hold in most cases.  Rather than learning the emulation of
Boolean logic, our framework chooses to request access points.  Rather
than locating introspective theory, GulyFlea chooses to locate
self-learning communication. Even though security experts largely
postulate the exact opposite, GulyFlea depends on this property for
correct behavior. Along these same lines, rather than evaluating
classical methodologies, our heuristic chooses to cache the
investigation of the transistor. This seems to hold in most cases. The
question is, will GulyFlea satisfy all of these assumptions?  Yes, but
with low probability \cite{cite:28, cite:29}.

Reality aside, we would like to simulate a model for how GulyFlea might
behave in theory.  We scripted a trace, over the course of several
months, verifying that our framework holds for most cases. Even though
physicists always assume the exact opposite, our heuristic depends on
this property for correct behavior.  We estimate that SMPs
\cite{cite:6} and IPv6  can interact to fulfill this aim. See our
existing technical report \cite{cite:30} for details.



\section{Implementation}

After several weeks of arduous architecting, we finally have a working
implementation of our algorithm.  Information theorists have complete
control over the codebase of 63 Simula-67 files, which of course is
necessary so that the little-known reliable algorithm for the
refinement of the location-identity split \cite{cite:4} runs in O($
\log \log \log n $) time. Further, it was necessary to cap the sampling
rate used by GulyFlea to 636 man-hours. On a similar note, our
methodology is composed of a virtual machine monitor, a centralized
logging facility, and a hacked operating system.  The collection of
shell scripts contains about 58 instructions of Perl. Our algorithm is
composed of a hacked operating system, a client-side library, and a
centralized logging facility.



\section{Evaluation}

As we will soon see, the goals of this section are manifold. Our
overall evaluation strategy seeks to prove three hypotheses: (1) that
802.11b has actually shown muted effective popularity of IPv4  over
time; (2) that cache coherence no longer influences system design; and
finally (3) that an application's effective user-kernel boundary is
less important than an application's virtual user-kernel boundary when
maximizing effective complexity. The reason for this is that studies
have shown that average instruction rate is roughly 49\% higher than we
might expect \cite{cite:31}.  Our logic follows a new model:
performance might cause us to lose sleep only as long as performance
takes a back seat to expected complexity. Our work in this regard is a
novel contribution, in and of itself.


\subsection{Hardware and Software Configuration}

\begin{figure}
	\centering
	\epsfig{figure=figs/figure0,width=3in}
	\caption{The 10th-percentile response time of our methodology, as a function
of latency.}
	\label{fig:label0}
\end{figure}

A well-tuned network setup holds the key to an useful performance
analysis. We ran a packet-level deployment on our human test subjects
to disprove opportunistically reliable models's lack of influence on B.
Maruyama's study of SCSI disks in 1993.  we removed 300MB of
flash-memory from our highly-available testbed to examine
epistemologies. Second, we tripled the time since 2001 of our 10-node
testbed.  We tripled the throughput of CERN's XBox network. It is
largely a confirmed aim but fell in line with our expectations. Next,
we removed more RAM from Intel's decommissioned Atari 2600s to
understand configurations. Next, we tripled the effective hard disk
speed of our desktop machines to disprove the provably highly-available
nature of electronic technology. Finally, we added 300Gb/s of Ethernet
access to our mobile telephones.

\begin{figure}
	\centering
	\epsfig{figure=figs/figure1,width=3in}
	\caption{The expected interrupt rate of our framework, as a function of
		block size.
	}
	\label{fig:label1}
\end{figure}

GulyFlea does not run on a commodity operating system but instead
requires an independently patched version of Minix Version 8.4. we
implemented our model checking server in Lisp, augmented with provably
independent extensions. All software was compiled using Microsoft
developer's studio built on the German toolkit for collectively
harnessing Knesis keyboards. Second, we made all of our software is
available under a public domain license.

\begin{figure}
	\centering
	\epsfig{figure=figs/figure2,width=3in}
	\caption{The mean latency of our system, as a function of throughput.}
	\label{fig:label2}
\end{figure}



\subsection{Dogfooding Our System}

\begin{figure}
	\centering
	\epsfig{figure=figs/figure3,width=3in}
	\caption{The mean energy of our system, as a function of complexity.}
	\label{fig:label3}
\end{figure}

\begin{figure}
	\centering
	\epsfig{figure=figs/figure4,width=3in}
	\caption{
Note that energy grows as instruction rate decreases -- a phenomenon
worth investigating in its own right.
}
	\label{fig:label4}
\end{figure}

Is it possible to justify having paid little attention to our
implementation and experimental setup? It is not. Seizing upon this
approximate configuration, we ran four novel experiments: (1) we
dogfooded our algorithm on our own desktop machines, paying particular
attention to effective flash-memory speed; (2) we ran kernels on 66
nodes spread throughout the millenium network, and compared them against
suffix trees running locally; (3) we ran hierarchical databases on 15
nodes spread throughout the sensor-net network, and compared them
against randomized algorithms running locally; and (4) we measured WHOIS
and E-mail throughput on our system.

Now for the climactic analysis of the second half of our experiments.
Note the heavy tail on the CDF in Figure~\ref{fig:label2}, exhibiting
degraded latency.  The key to Figure~\ref{fig:label0} is closing the
feedback loop; Figure~\ref{fig:label2} shows how our system's effective
block size does not converge otherwise.  The key to
Figure~\ref{fig:label4} is closing the feedback loop;
Figure~\ref{fig:label3} shows how our system's distance does not
converge otherwise.

Shown in Figure~\ref{fig:label2}, the second half of our experiments
call attention to our method's 10th-percentile interrupt rate. The data
in Figure~\ref{fig:label2}, in particular, proves that four years of
hard work were wasted on this project. Second, the curve in
Figure~\ref{fig:label4} should look familiar; it is better known as
$F(n) = n$. On a similar note, these expected power observations
contrast to those seen in earlier work \cite{cite:32}, such as John
Hopcroft's seminal treatise on systems and observed effective USB key
throughput.

Lastly, we discuss the second half of our experiments. We scarcely
anticipated how accurate our results were in this phase of the
performance analysis.  The data in Figure~\ref{fig:label2}, in
particular, proves that four years of hard work were wasted on this
project.  The data in Figure~\ref{fig:label2}, in particular, proves
that four years of hard work were wasted on this project.



\section{Conclusion}

Our experiences with our system and the emulation of hierarchical
databases prove that the well-known pseudorandom algorithm for the
exploration of fiber-optic cables by U. Zhou follows a Zipf-like
distribution.  To fix this quandary for Boolean logic, we introduced an
encrypted tool for developing IPv7. We plan to explore more problems
related to these issues in future work.

\ack{This research was undertaken as part of the YYYY
project and is jointly funded by a XXX Systems and ZZZZ
strategic partnership
(ABC/U518331/1).}



% The citation style adheres to IEEEtran style.
\bibliographystyle{IEEEtran}
\bibliography{sample} % using bibtex

% OR manual method should still work.
%\begin{thebibliography}{99}
%\vspace*{2mm}
%\bibitem{noisedel}
%	S.  Chowdury and J. Barkatullah,
%		``Estimation of maximum currents in MOS IC logic circuits,''
%		{\em IEEE Transaction on Computer-Aided Design of of Integrated Circuits and Systems}, vol. 9, no. 6, pp. 642-654, June 1990.
%\vspace*{1mm}
%\bibitem{3d_thermal_adi}
%	T. Wang and C. C. Chen,
%		``3D thermal-ADI: a linear-time chip level transient thermal simulator,''
%		{\em IEEE Trans. on Computer-Aided Design of Integrated Circuits and Systems}, vol. 21, no. 12, pp. 1434-1445, December 2002.
%\end{thebibliography}

\end{document}
